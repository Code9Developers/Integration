\documentclass{article}
 \usepackage{hyperref}
\usepackage{graphicx}

\begin{document}
    
    \begin{center}
        \includegraphics{Graphics/uplogo.jpg}
        
        {\Huge 
        2017 COS 301 Project \linebreak
        Coding Standards \linebreak 
        \par}
        
        {\Huge
        TEAM CODE 9
        \linebreak
        \par}
        
        \begin{LARGE}
            Seonin David
            \linebreak
            \linebreak
            Joshua Moodley
            \linebreak
            \linebreak
            Jaques Smulders
            \linebreak
            \linebreak
            Jordan Daubinet
            \linebreak
            \linebreak
            Nicaedin Suklul
        \end{LARGE}
    \end{center}
    
    \begin{figure}[b]
    \centering
        \includegraphics[width=8cm]{Graphics/kpmgLogo.jpg}
    \end{figure}
    
    \newpage
    
    \section{Introduction}
        \begin{flushleft}
        This is a document defining the coding standards for the kpmg's project management system, it defines the rules which govern how the development team members are to code. This includes comments structure, method structure and class structure aswell as the variable naming conventions.
        \end{flushleft}
        
    \section{Classes}
        \begin{flushleft}
            A Class header is to be included with every class developed.
            The class header is to have the following structure: A brief description of the classes purpose, Author name, an update History in the form of version numbers, a list of methods which have been tested and a list of methods which have yet to be tested.
        \end{flushleft}
    
    \section{Naming Conventions}
        \begin{flushleft}
            \begin{itemize}
                \item Class Names: are to begin with capital letters and have no spaces or numeric numbers or any symbols. Example "Shape".
                \item Method Names: should use a camel case style whereby each name begins with a lower case letter and each new word included in the name begins with an upper case letter. Example "getLength()".
                \item Variable Names: should be all under case with a lower score between describing words in the name. Example "the\_height"
            \end{itemize} 
        \end{flushleft}
    
    \section{Methods}
        \begin{flushleft}
            All methods should begin with a commented section describing the purpose of the method. In this section the inputs requered by the method and the ouput given by the method should be described. 
        \end{flushleft}
    \section{Comments}
        \begin{flushleft}
            All comments should be multi-line comments starting with "/*" and ending with "*/". There should be no comments within methods, only the methos description before and if needed an extra section after the method which can be used to explain addtional information about the method.
        \end{flushleft}
        
        \newpage
    \section{Formatting conventions}
        \begin{flushleft}
            For classes and methods, the curly bracket should start as a new line after the class or method definition and end on a new line following the last line of code for the class or method. Methods should be seperated with a new line.
        \end{flushleft}
\end{document}
